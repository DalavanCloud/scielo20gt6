\documentclass[utf8]{beamer}

\usepackage[T1]{fontenc}
\usepackage[american]{babel}
\usepackage{tabu}

\mode<presentation>
\usetheme{Warsaw}
\usecolortheme{beaver}
\beamertemplatenavigationsymbolsempty

\setbeamertemplate{footline}{\leavevmode\hbox{%
  \begin{beamercolorbox}[wd=.35\paperwidth, ht=2.25ex, dp=1ex, center]
                        {author in head/foot}
    \usebeamerfont{author in head/foot}
      \insertshorttitle
  \end{beamercolorbox}%
  \begin{beamercolorbox}[wd=.65\paperwidth, ht=2.25ex, dp=1ex, center]
                        {title in head/foot}
    \usebeamerfont{title in head/foot}
      \insertshortsubtitle \hfill
      \insertframenumber\,/\,\inserttotalframenumber
  \end{beamercolorbox}%
}}

\title{WG6 Report}
\subtitle{SciELO 20 Years}
\date{2018-09-25}
\author{Danilo J. S. Bellini}


\begin{document}


\begin{frame}
  \begin{beamercolorbox}[sep=8pt, center, rounded=true, shadow=true]{title}%
    \usebeamerfont{title}\usebeamercolor[fg]{title}%
      \inserttitle\par%
    \vskip0.25em%
    \usebeamerfont{subtitle}\usebeamercolor[fg]{subtitle}%
      \insertsubtitle\par%
  \end{beamercolorbox}
  \vfill
  \begin{tabu} to \textwidth {rX}
    \emph{WG6 presentation title:} &
      Workshop on the use of data from the SciELO database \\
    \\
    \emph{Lecturer/speaker/rapporteur:} &
      Danilo J. S. Bellini \\
    \emph{Group coordinator:} &
      Gustavo Fonseca \\
    \emph{Executive secretary:} &
      Carolina Tanigushi \\
    \\
    \emph{WG6 date:} &
      2018-09-24 \\
    \emph{Report date:} &
      2018-09-25 \\
    \\
    \emph{Venue:} &
      Tivoli Mofarrej São Paulo Hotel \\
  \end{tabu}
\end{frame}


\begin{frame}{Summary}
  During the workshop, these had been done:

  \begin{itemize}
    \item Brief introduction to the data analysis processes,
          emphasizing data access, data cleaning
          and exploratory data analysis
    \item \emph{Hands-on} examples using Python and R,
          with emphasis in the \emph{Pandas} library resources,
          showing how data munging, normalization, data visualization
          and other data analysis processes can be performed
    \item Explanation of data analyses previously performed
          on data coming from SciELO and external sources
  \end{itemize}

  Most of the time was spent on exploratory data analysis,
  interpretation of descriptive statistics and visualization.
  Some highlights of what had been studied in more depth include:

  \begin{itemize}
    \item Hirsch index
    \begin{itemize}
      \item Google Scholar's \texttt{h5-index} and \texttt{h5-median}
      \item Calculation from raw Dimensions' data
      \item SCImagoJR's H index
    \end{itemize}
    \item Field Citation Ratio (FCR) from Dimensions
  \end{itemize}

\end{frame}


\begin{frame}{Tools}
  Two programming languages had been used,
  besides several:
  \begin{itemize}
    \item Python
    \begin{itemize}
      \item IPython / Jupyter Notebook
      \item Python built-in modules
            (\texttt{csv}, \texttt{statistics}, \texttt{urllib},
             \texttt{json}, \texttt{glob}, \texttt{os}, \texttt{re},
             \texttt{collections}, \texttt{itertools},
             \texttt{pprint})
      \item \texttt{numpy}
      \item \texttt{pandas}
      \item \texttt{matplotlib}
      \item \texttt{seaborn}
      \item \texttt{openpyxl},
            to open XLSX files
      \item \texttt{scipy.stats},
            to calculate the Pearson's correlation coefficient
      \item NetworkX,
            a graph manipulation library
            including an API to draw graphs with matplotlib
    \end{itemize}

    \item R
    \begin{itemize}
      \item R built-in modules
            (\texttt{base}, \texttt{utils}, \texttt{stats},
             \texttt{graphics})
      \item R Studio, an IDE for R, for creating R Markdown notebooks
      \item \texttt{dplyr},
            to perform grouping operations
            similar to SQL's \texttt{GROUP BY}
            and Pandas' \texttt{DataFrame.groupby}
    \end{itemize}
  \end{itemize}
\end{frame}


\begin{frame}{Data sources}
  Several analyses were performed before the workshop,
  whose processes and results were part of it.
  The data that had been studied in every analysis
  performed during and before the workshop came from
  these sources:

  \begin{itemize}
    \item SciELO's JSON APIs (RESTful) from:
    \begin{itemize}
      \item \emph{ArticleMeta}, to get journal metadata
      \item \emph{Ratchet}, to get access data
    \end{itemize}
    \item SciELO's \texttt{articlemeta} Python library,
          an alternative way to access the ArticleMeta API
    \item Reports from the \emph{SciELO Analytics}
    \item \emph{SciELO Citation Index} entries from
          the \emph{Web of Science}
    \item \emph{Dimensions} data regarding two journals:
          \emph{Nauplius} and
          \emph{Brazilian Journal of Plant Physiology}
    \item \emph{SCImagoJR}'s CSV with all SJR and H indices for 2017
    \item \emph{Scopus}' XLSX with all the data they make available
  \end{itemize}

\end{frame}


\begin{frame}{Introduction to the analysis}

  Besides:

  \begin{itemize}
    \item Identifying which collections have data in SciELO analytics
          (all certified and development collections,
           besides the active independent collections)
    \item Downloading all SciELO analytics reports
    \item Evaluating if the network reports
          have everything from the remaining reports
    \item Simplifying the column names
    \item Normalizing/cleaning the ISSN
          when dealing with multiple collections
    \item Normalizing the thematic area
          (dealing with unfilled data)
  \end{itemize}

  It had been seen how to plot data,
  with a strong emphasis on data interpretation
  and multiple types of plots
  (bar plots, line plots, box-and-whisker plots, heat maps,
   scatterplots, etc.),
  as well as subplot splitting/grouping.

\end{frame}


\begin{frame}{Previously prepared analyses}
  \begin{itemize}
    \item Number of indexed journals in the SciELO network
    \item Deindexing reason in the SciELO Brazil collection
    \item Evaluating the daily access in the SciELO Brazil collection
    \item Three indices in Scopus 2017: CiteScore, SNIP and SJR
    \item SCImago Journal Rank in 2017, including SJR and H index
    \item FCR and H index in Dimensions
    \item Google Scholar indices
    \item Languages of research articles in SciELO Brazil,
          by thematic area,
             document publication year and
             journal indexing year
    \item Citations in the SciELO CI
    \item Proportion of Brazil in affiliation institutions
          in research articles
          from journals in the SciELO Brazil collection
  \end{itemize}
\end{frame}


\begin{frame}{Results}
  \begin{itemize}
    \item The proportion of Brazilian affiliations
          of research articles in the SciELO Brazil collection
          is decreasing
    \item Most citations of research articles in the SciELO network
          come from documents/journals
          that aren't in the SciELO network.
          For research articles written in English,
          $76\%$ of the received citations
          comes from documents external to the SciELO network
    \item The normalization step when calculating
          the FCR and its non-standard average calculation
          can easily \emph{push down} the result
          (e.g. a journal with $10$ documents receiving $15$ citations
           and $3$ documents with zero citations
           would have an average of citations of less than $7.5$,
           before this number gets normalized
           by the year and field of research),
          making it an index best fit to evaluate older journals
          that are no longer publishing
    \item We should always look for
          the mathematics that defines an index,
          as that evaluation can already give us some insights
          regarding its bias towards some documents/journals
  \end{itemize}
\end{frame}


\begin{frame}{Results}
  \begin{itemize}
    \item Scopus indices should be taken with care:
          mixing the data from all countries
          makes it hard to compare data from SciELO and
          from other journals
    \item In SciELO Brazil, $95\%$ of the journals
          marked as \emph{deceased}
          were actually just \emph{renamed}
          to a new journal title/ISSN
    \item Matching data with external sources is difficult
          without a common standardized index
          such as the ISSN and DOI
    \item $0.8\%$ of SciELO journals are marked in Scopus as not open,
          which seem to be an issue regarding Scopus data
  \end{itemize}
\end{frame}


\end{document}
